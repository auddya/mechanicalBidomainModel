\documentclass[11pt]{article}
\usepackage[
a4paper,
margin=1in,
headsep=4pt, % separation between header rule and text
]{geometry}
\usepackage{xcolor}
\usepackage{fancyhdr}
\usepackage{tgschola}
\usepackage{lastpage}
\usepackage[natbibapa]{apacite}
\usepackage{listings}
\usepackage{subfigure}
\usepackage{color}
\usepackage{dsfont}
\usepackage{footmisc}
\usepackage{verbatim}
\usepackage{smartdiagram}
\setlength{\marginparwidth}{0cm}
\setlength{\topmargin}{0cm}
\setlength{\voffset}{0cm}
\setlength{\headsep}{0cm}
\definecolor{dkgreen}{rgb}{0,0.6,0}
\definecolor{gray}{rgb}{0.5,0.5,0.5}
\definecolor{mauve}{rgb}{0.58,0,0.82}
\usepackage[utf8]{inputenc}
\graphicspath{{/Users/auddya/uw-me/2018/mechanicalBidomainModel/}}
\lstset{frame=tb,
	language=Java,
	aboveskip=3mm,
	belowskip=3mm,
	showstringspaces=false,
	columns=flexible,
	basicstyle={\small\ttfamily},
	numbers=none,
	numberstyle=\tiny\color{gray},
	keywordstyle=\color{blue},
	commentstyle=\color{dkgreen},
	stringstyle=\color{mauve},
	breaklines=true,
	breakatwhitespace=true,
	tabsize=3
}
\pagestyle{fancy}
\fancyhf{}
\fancyhead[C]{%
	\footnotesize\sffamily
	\yourname\quad
	web: \textcolor{blue}{\itshape\yourweb}\quad
	\textcolor{blue}{\youremail}}
\fancyfoot[C]{Page \thepage\ of \pageref{LastPage}}

\newcommand{\soptitle}{A computational study of mechanical bidomain model in durotaxis}

\newcommand{\yourname}{Debabrata Auddya}
\newcommand{\youremail}{auddya@wisc.edu}
\newcommand{\yourweb}{https://github.com/auddya}

\newcommand{\statement}[1]{\par\medskip
	\textcolor{blue}{\textbf{#1:}}\space
}

\usepackage[
breaklinks,
pdftitle={\yourname - \soptitle},
pdfauthor={\yourname},
unicode
]{hyperref}

\begin{document}
	
	\begin{center}
		\Large\soptitle
	\end{center}

\section*{Parameters}
N = 101 (No of nodes) \\
L = 0.005 m (Length of domain) \\
nu = 1000 Pa (Intracellular modulus) \\
mu\_zero = 1000 Pa (Extracellular modulus) \\
K = 50000000000 Pa/$m^{2}$ (Stiffness) \\
T =  200 Pa (Tension) \\
\section*{Code and Results}
\begin{lstlisting}
N = 101;
L = 0.005; %0.005m
g = 0; %100000; %100000 Pa/m
mu_zero = 1000; %1000 Pa
nu = 1000; %1000 Pa
K = 50000000000; %50GPa/m2
T = 200; %200Pa
w = zeros(N,1); %Extracellular displacement
u = zeros(N,1); %Intracellular displacement
x = zeros(N,1); %x position, useful when plotting
delta = (2*L)/(N-1); %Spacing along x direction
iterations = 100;
for i = 1:N
x(i) = L*(2*(i-1)/(N-1)-1); 
mu(i) = mu_zero + g*x(i);
end
for k = 1:iterations
for i = 2:(N-1)
a(i) = 4*mu(i)*(w(i+1)+w(i-1))+(mu(i+1)-mu(i-1))*(w(i+1)-w(i-1));
b(i) = 4*nu*(u(i+1)+u(i-1));
A(i) = 8*mu(i) + K*delta*delta;
C = 8*nu + K*delta*delta;
B = K*delta*delta;
u(i) = (a(i)*B + A(i)*b(i))/(A(i)*C - B*B);
w(i) = (a(i)/A(i)) + (B/A(i))*((a(i)*B + A(i)*b(i))/(A(i)*C - B*B));
end
%Apply Boundary Conditions
u(1) = u(2) + (T*delta/(4*nu));
w(1) = w(2);
u(N) = u(N-1) - (T*delta/(4*nu));
w(N) = w(N-1);
end 
for i = 1:N
h(i) = u(i)-w(i);
end 
plot(x,h)  %if you want plot with x in mm, use plot(x*1000,h)
xlabel('x');
ylabel('u-w');
title('Difference between extracellular and intracellular displacement');
\end{lstlisting}
\begin{figure}[htb]
	\begin{center}
		%
		\subfigure[Difference in displacements for g=0]{%
			\label{fig:uw0}
			\includegraphics[width=0.4\textwidth]{/testResults/u-w(g_0).jpg}
		}%
		\subfigure[Difference in displacements for g=$10^5$]{%
			\label{fig:uwvar}
			\includegraphics[width=0.4\textwidth]{/testResults/u-w(g_var).jpg}
		}\\ %  ------- End of the first row ----------------------%
		\subfigure[Intracellular displacement for g=0]{%
			\label{fig:u0}
			\includegraphics[width=0.4\textwidth]{/testResults/u(g_0).jpg}
		}%
		\subfigure[Intracellular displacement for g=$10^5$]{%
			\label{fig:uvar}
			\includegraphics[width=0.4\textwidth]{/testResults/u(g_var).jpg}
		}\\
			\subfigure[Extracellular displacement for g=0]{%
		\label{fig:w0}
		\includegraphics[width=0.4\textwidth]{/testResults/w(g_0).jpg}
	  }%
	\subfigure[Extracellular displacement for g=$10^5$]{%
		\label{fig:wvar}
		\includegraphics[width=0.4\textwidth]{/testResults/w(g_var).jpg}
	 }%
	\end{center}
	\caption{%
Comparison of results for g=0 and g=$10^5$ Pa/m (Iteration: 100)
	}%
	\label{fig:subfigures}
\end{figure}
\begin{figure}[htb]
	\begin{center}
		%
		\subfigure[Difference in displacements for g=0]{%
			\label{fig:uw02}
			\includegraphics[width=0.4\textwidth]{/testResults/u-w2(g_0).jpg}
		}%
		\subfigure[Difference in displacements for g=$10^5$]{%
			\label{fig:uwvar2}
			\includegraphics[width=0.4\textwidth]{/testResults/u-w2(g_var).jpg}
		}\\ %  ------- End of the first row ----------------------%
		\subfigure[Intracellular displacement for g=0]{%
			\label{fig:u02}
			\includegraphics[width=0.4\textwidth]{/testResults/u2(g_0).jpg}
		}%
		\subfigure[Intracellular displacement for g=$10^5$]{%
			\label{fig:uvar2}
			\includegraphics[width=0.4\textwidth]{/testResults/u2(g_var).jpg}
		}\\
		\subfigure[Extracellular displacement for g=0]{%
			\label{fig:w02}
			\includegraphics[width=0.4\textwidth]{/testResults/w2(g_0).jpg}
		}%
		\subfigure[Extracellular displacement for g=$10^5$]{%
			\label{fig:wvar2}
			\includegraphics[width=0.4\textwidth]{/testResults/w2(g_var).jpg}
		}%
	\end{center}
	\caption{%
		Comparison of results for g=0 and g=$10^5$ Pa/m (Iteration: 1000, N = 1001)
	}%
	\label{fig:subfigures2}
\end{figure}
\end{document}
