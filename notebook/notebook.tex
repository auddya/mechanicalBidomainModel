\documentclass[a4paper,oneside,11pt]{report}
\usepackage[cm]{fullpage}
\usepackage{lmodern,amsmath,amssymb}
\usepackage{a4wide}
\setlength{\marginparwidth}{3cm}
\setlength{\topmargin}{0cm}
\setlength{\voffset}{0cm}
\setlength{\headsep}{0cm}
\title{Modeling Durotaxis using Mechanical Bidomain model}
\author{Prof Brad Roth, Debabrata Auddya}
%\usepackage{etoolbox}
%\preto\equation{\setcounter{equation}{0}}
%\makeatletter
%\pretocmd\start@gather{\setcounter{equation}{0}}{}{}
%\pretocmd\start@align{\setcounter{equation}{0}}{}{}
%\pretocmd\start@multline{\setcounter{equation}{0}}{}{}
%\makeatother
\usepackage{listings}
\usepackage{color}
\usepackage{dsfont}
\usepackage{footmisc}
\usepackage{verbatim}
\usepackage{smartdiagram}
\setlength{\marginparwidth}{0cm}
\setlength{\topmargin}{0cm}
\setlength{\voffset}{0cm}
\setlength{\headsep}{0cm}
\definecolor{dkgreen}{rgb}{0,0.6,0}
\definecolor{gray}{rgb}{0.5,0.5,0.5}
\definecolor{mauve}{rgb}{0.58,0,0.82}

\lstset{frame=tb,
	language=Java,
	aboveskip=3mm,
	belowskip=3mm,
	showstringspaces=false,
	columns=flexible,
	basicstyle={\small\ttfamily},
	numbers=none,
	numberstyle=\tiny\color{gray},
	keywordstyle=\color{blue},
	commentstyle=\color{dkgreen},
	stringstyle=\color{mauve},
	breaklines=true,
	breakatwhitespace=true,
	tabsize=3
}
\begin{document}
\maketitle
\section*{Analytical treatment - 10/03/2018}
The equations formulated below consider a gradient in the extracellular stiffness in a 1-d strand of tissue using the mechanical bidomain model. \\

Consider $u_{x}$ and $w_{x}$ as the intra- and extra-cellular displacements of the bidomain layer. x being the principal direction of the one-dimensional tissue strand, T being the tension, p and q being the intra- and extra-cellular pressure contributions arising mostly from hydrostatic forces. \\

The standard equations of the bidomain model are as follows \cite{roth2012}: 
\begin{align}
-\frac{\partial p}{\partial x} + \nu(\frac{\partial^{2}u_x}{\partial x^2} + \frac{\partial^{2}u_x}{\partial y^2} ) + \gamma\frac{\partial^{2}u_x}{\partial x^2} + \frac{\partial T}{\partial x} = K(u_x - w_x) \\
-\frac{\partial q}{\partial x} + \mu(\frac{\partial^{2}w_x}{\partial x^2} + \frac{\partial^{2}w_x}{\partial y^2} ) = -K(u_x - w_x) 
\end{align}   
For this problem we consider:
\begin{itemize} 
\item $\mu = \mu_0 + gx$ 
\item No contributions along the y axis
\item p,q and T are constant and hence their gradients are zero along the x axis
\item Slope of $\mu$ is a constant 
\item K is same for intra- and extra-cellular displacements
\end{itemize}
\begin{align}
    \nu\frac{\partial^{2}u_x}{\partial x^2} + \gamma\frac{\partial^{2}u_x}{\partial x^2} = K(u_x - w_x)  \\
    \mu_0\frac{\partial^{2}w_x}{\partial x^2} + gx\frac{\partial^{2}w_x}{\partial x^2} = -K(u_x - w_x) 
\end{align}
Simplifying the above expression, replacing $u_x$ in equation (2) from (1): 
\begin{align}
   u_x = w_x + \frac{\nu}{K}\frac{\partial^{2}u_x}{\partial x^2} + \frac{\gamma}{K}\frac{\partial^{2}u_x}{\partial x^2} + \gamma\frac{\partial^{2}u_x}{\partial x^2} \\
   \therefore \frac{\mu_0}{K}\frac{\partial^{2}w_x}{\partial x^2} + x\frac{g}{k}\frac{\partial^{2}w_x}{\partial x^2} = - ( \frac{\nu}{K}\frac{\partial^{2}u_x}{\partial x^2} + \frac{\gamma}{K}\frac{\partial^{2}u_x}{\partial x^2} )
\end{align}
Final form the equation can be written as: 
\begin{align}
\frac{\partial^{2}w_x}{\partial x^2} (\frac{\mu_0}{K} + \frac{gx}{K}) + \frac{\partial^{2}u_x}{\partial x^2} (\frac{\nu}{K} + \frac{\gamma}{K} ) = 0
\end{align}
\clearpage
\section*{Analytical treatment - 10/04/2018}
The equations formulated below consider a gradient in the extracellular stiffness in a 1-d strand of tissue using the mechanical bidomain model. \\

Consider $u_{x}$ and $w_{x}$ as the intra- and extra-cellular displacements of the bidomain layer. x being the principal direction of the one-dimensional tissue strand, T being the tension, p and q being the intra- and extra-cellular pressure contributions arising mostly from hydrostatic forces. \\

The intra and extra-cellular stresses arising in the 1-d strand of tissue can be expressed as \cite{roth2015}: 
\begin{align}
\tau_{ix} = -p + 2\nu\epsilon_{ix} + T \\
\tau_{ex} = -q + 2\mu(x)\epsilon_{ex}   
\end{align}
Relationship between the strains and displacement can be written as: 
\begin{align}
\frac{\partial\tau_{ix}}{\partial x} = K (u_x - w_x) \\
\frac{\partial\tau_{ex}}{\partial x} = - K (u_x - w_x) 
\end{align}
Using equation (8) and (9) in (10) and (11) the resulting intra- and extra-cellular equations are: 
\begin{align}
\frac{\partial}{\partial x}(-p + 2\nu\frac{\partial u_x}{\partial x}) = K (u_x - w_x) \\
\frac{\partial}{\partial x}(-q + 2\mu (x) \frac{\partial w_x}{\partial x}) = -K (u_x - w_x)
\end{align}
For the problem we have assumed $\mu = \mu_0 + gx$ , where g is a constant. Rewriting (12) and (13) we have:
\begin{align}
- \frac{\partial p}{\partial x} + 2\nu\frac{\partial^{2}u_x}{\partial x^2} = K (u_x - w_x) \\
- \frac{\partial q}{\partial x} + 2(g\frac{\partial w_x}{\partial x} + \frac{\partial^{2}w_x}{\partial x^2} \mu(x)) = -K(u_x - w_x)
\end{align}
\clearpage
\section*{Analytical treatment - 11/13/2018}
In order to obtain an analytical solution for the problem initially we consider g = 0. The boundary limits of the 1-dimensional problem assume the length of the domain spanning form x=-L to x=+L. The stresses are taken to be zero at each of the boundaries. We begin by implementing a trial solution for $u_x$ and $w_x$.
\begin{align}
u_x = Ax + Bsinh \Big ( \frac{x}{\sigma} \Big )\\
w_x = Cx + Dsinh \Big ( \frac{x}{\sigma} \Big )
\end{align}
For estimating the values of A,B,C,D we assume additionally p=q=0. 
Rewriting (12):
\begin{align}
2\nu\frac{\partial^{2}}{\partial x^2}\Big ( Ax + Bsinh\Big ( \frac{x}{\sigma} \Big ) \Big ) = K \Big ( Ax + Bsinh \Big ( \frac{x}{\sigma} \Big ) - Cx - Dsinh \Big ( \frac{x}{\sigma} \Big )  \Big )\\
2\nu\frac{\partial}{\partial x} \Big ( A + \frac{B}{\sigma}cosh\Big ( \frac{x}{\sigma} \Big ) \Big ) = K \Big ((A - C)x + (B - D)sinh\Big ( \frac{x}{\sigma} \Big ) )\\
2\nu\frac{B}{\sigma^2} sinh\Big ( \frac{x}{\sigma} \Big ) = K \Big ( (A - C)x + (B - D)sinh\Big ( \frac{x}{\sigma} \Big )\Big )
\end{align}
and (13)
\begin{align}
2\mu(x)\frac{\partial^{2}}{\partial x^2}\Big ( Cx + Dsinh\Big ( \frac{x}{\sigma} \Big ) \Big ) = -K \Big ( Ax + Bsinh \Big ( \frac{x}{\sigma} \Big ) - Cx - Dsinh \Big ( \frac{x}{\sigma} \Big )  \Big )\\
2\mu(x)\frac{\partial}{\partial x} \Big ( C + \frac{D}{\sigma}cosh\Big ( \frac{x}{\sigma} \Big ) \Big ) = -K \Big ((A - C)x + (B - D)sinh\Big ( \frac{x}{\sigma} \Big ) )\\
2\mu(x)\frac{D}{\sigma^2} sinh\Big ( \frac{x}{\sigma} \Big ) = -K \Big ( (A - C)x + (B - D)sinh\Big ( \frac{x}{\sigma} \Big )\Big )
\end{align}
From (20) and (23) we have:
\begin{align}
D = -\frac{\nu}{\mu(x)} B
\end{align}
Comparing coefficients of hyperbolic sine terms we see that (20) and (23) is satisfied only when the coefficient of the linear term is zero. Therefore,
\begin{align}
A = C
\end{align}
Hence the length constant\footnote{This might also be a variable since $\mu$ is a function of x for the problem} $\sigma$ has the value\footnote{$\mu(x)$ is also written as $\mu$}
\begin{equation}
\begin{aligned}
\sigma = \sqrt{\frac{2\nu\mu}{K(\nu + \mu)}}
\end{aligned}
\end{equation}
To obtain the values of the unknown parameters B and C, we impose boundary conditions. At the edge (x = $\pm$L) we have, normal stresses $\tau_{ix}$ and $\tau_{ex}$ as zero. As a result we get,
\begin{align}
C + \frac{D}{\sigma}cosh\Big ( \frac{L}{\sigma} \Big ) = 0\\
A + \frac{B}{\sigma}cosh\Big ( \frac{L}{\sigma} \Big ) = -\frac{T}{2\nu}
\end{align} 
Solving (27) and (28) using (24)-(26) we have:
\begin{align}
A = C = -\frac{T}{2(\nu + \mu)}\\
B = -\frac{T}{2\nu}\Big ( \frac{\mu}{\nu + \mu} \Big )\frac{\sigma}{cosh(\frac{L}{\sigma})}\
, D = \frac{T}{2(\mu + \nu)}\frac{\sigma}{cosh(\frac{L}{\sigma})}
\end{align}
Using this we get the intra and extracellular displacement in terms of T as:
\begin{align}
u_x = -\frac{T}{2(\nu + \mu)} \Big ( x + \frac{\mu}{\nu} \sigma \frac{sinh(\frac{x}{\sigma})}{cosh(\frac{L}{\sigma})}\Big )\\
w_x = -\frac{T}{2(\nu + \mu)} \Big ( x - \sigma \frac{sinh(\frac{x}{\sigma})}{cosh(\frac{L}{\sigma})}\Big )
\end{align}
Since $\mu$ is variable for the problem we now write it as: $\mu = \mu_0 + gx$. Remaining derivation of $u_x$ and $w_x$ along with its derivatives and double derivatives has been attached separately. 
\newpage
\section*{Analytical treatment - 12/22/2018}
We are considering plane stress conditions for the model. Nothing depends on y. 
\begin{align*}
\tau_{exx} = -q + 2\mu\epsilon_{exx} \\
\tau_{eyy} = -q \\
\tau_{ezz} = -q + 2\mu\epsilon_{ezz} \\
\end{align*}
Since $\tau_{ezz}$  = 0 so
\begin{align*}
-q + 2\mu\epsilon_{ezz} = 0
\end{align*}
or, $q=2\mu\epsilon_{ezz}$, but, $\epsilon_{exx} + \epsilon_{ezz} = 0$, so 
\begin{align*}
q=-2\mu\epsilon_{exx}
\end{align*}
Therefore the stresses along the principal directions can be written as:
\begin{align*}
\tau_{exx} = 4\mu\epsilon_{exx} \hspace{0.5cm} \tau_{eyy} = 2\mu\epsilon_{exx}\hspace{0.5cm} \tau_{ezz}=0\hspace{0.5cm}\tau_{exy}=0
\end{align*}
The final partial differential form of the equations look like: 
\begin{align*}
\frac{\partial\tau_{exx}}{\partial x} + \frac{\partial\tau_{exy}}{\partial y} = -K(u_x - w_x) \\
or, \hspace{0.5cm}4\Big ( \frac{\partial \mu}{\partial x}\epsilon_{exx}  + \mu \frac{\partial \epsilon_{exx}}{\partial x}\Big ) = -K(u_x - w_x)
\end{align*}
For the intracellular layer the principal stress is the same except that Tension T along the principal direction (x) is also taken into account. Thus the final set of working equations are, considering $\mu$ has a gradient along x and $\nu$ is a constant for the problem. 
\[\boxed{\!\begin{aligned}
	&4\mu\frac{\partial^2 w}{\partial x^2} + 4\frac{\partial w}{\partial x}\frac{\partial \mu}{\partial x} = -K(u_x - w_x) \\
 &4\nu\frac{\partial^2 u}{\partial x^2} = K(u_x - w_x)
\end{aligned}
}
\]
\section*{$\mu$ = constant }
\[\boxed{\!\begin{aligned}
	&4\mu\frac{\partial^2 w}{\partial x^2} = -K(u_x - w_x) \\
	&4\nu\frac{\partial^2 u}{\partial x^2} = K(u_x - w_x)
	\end{aligned}
}
\]
The stresses are taken to be zero at each of the boundaries. We begin by implementing a trial solution for $u_x$ and $w_x$.
\begin{align*}
u_x = Ax + Bsinh \Big ( \frac{x}{\sigma} \Big )\\
w_x = Cx + Dsinh \Big ( \frac{x}{\sigma} \Big )
\end{align*}
For estimating the values of A,B,C,D we assume: 
Rewriting (12):
\begin{align*}
4\nu\frac{\partial^{2}}{\partial x^2}\Big ( Ax + Bsinh\Big ( \frac{x}{\sigma} \Big ) \Big ) = K \Big ( Ax + Bsinh \Big ( \frac{x}{\sigma} \Big ) - Cx - Dsinh \Big ( \frac{x}{\sigma} \Big )  \Big )\\
4\nu\frac{\partial}{\partial x} \Big ( A + \frac{B}{\sigma}cosh\Big ( \frac{x}{\sigma} \Big ) \Big ) = K \Big ((A - C)x + (B - D)sinh\Big ( \frac{x}{\sigma} \Big ) )\\
4\nu\frac{B}{\sigma^2} sinh\Big ( \frac{x}{\sigma} \Big ) = K \Big ( (A - C)x + (B - D)sinh\Big ( \frac{x}{\sigma} \Big )\Big )
\end{align*}
and (13)
\begin{align*}
4\mu(x)\frac{\partial^{2}}{\partial x^2}\Big ( Cx + Dsinh\Big ( \frac{x}{\sigma} \Big ) \Big ) = -K \Big ( Ax + Bsinh \Big ( \frac{x}{\sigma} \Big ) - Cx - Dsinh \Big ( \frac{x}{\sigma} \Big )  \Big )\\
4\mu(x)\frac{\partial}{\partial x} \Big ( C + \frac{D}{\sigma}cosh\Big ( \frac{x}{\sigma} \Big ) \Big ) = -K \Big ((A - C)x + (B - D)sinh\Big ( \frac{x}{\sigma} \Big ) )\\
4\mu(x)\frac{D}{\sigma^2} sinh\Big ( \frac{x}{\sigma} \Big ) = -K \Big ( (A - C)x + (B - D)sinh\Big ( \frac{x}{\sigma} \Big )\Big )
\end{align*}
From (20) and (23) we have:
\begin{align}
D = -\frac{\nu}{\mu} B
\end{align}
Comparing coefficients of hyperbolic sine terms we see that (20) and (23) is satisfied only when the coefficient of the linear term is zero. Therefore,
\begin{align}
A = C
\end{align}
Length constant $\sigma$ remains as: 
\begin{align}
	\sigma = \sqrt{\frac{4\nu\mu}{K(\nu + \mu)}}
\end{align}
To obtain the values of the unknown parameters B and C, we impose boundary conditions. At the edge (x = $\pm$L) we have, normal stresses $\tau_{ix}$ and $\tau_{ex}$ as zero. As a result we get,
\begin{align}
C + \frac{D}{\sigma}cosh\Big ( \frac{L}{\sigma} \Big ) = 0\\
A + \frac{B}{\sigma}cosh\Big ( \frac{L}{\sigma} \Big ) = -\frac{T}{4\nu}
\end{align}
Solving the equations using given by (33) - (35) and using them in (36) and (37) we have: 
Solving (27) and (28) using (24)-(26) we have:
\begin{align}
A = C = -\frac{T}{4(\nu + \mu)}\\
B = -\frac{T}{4\nu}\Big ( \frac{\mu}{\nu + \mu} \Big )\frac{\sigma}{cosh(\frac{L}{\sigma})}\
, D = \frac{T}{4(\mu + \nu)}\frac{\sigma}{cosh(\frac{L}{\sigma})}
\end{align}
Using this we get the intra and extracellular displacement in terms of T as:
\begin{align}
u_x = -\frac{T}{4(\nu + \mu)} \Big ( x + \frac{\mu}{\nu} \sigma \frac{sinh(\frac{x}{\sigma})}{cosh(\frac{L}{\sigma})}\Big )\\
w_x = -\frac{T}{4(\nu + \mu)} \Big ( x - \sigma \frac{sinh(\frac{x}{\sigma})}{cosh(\frac{L}{\sigma})}\Big )
\end{align}
\section*{$\mu = \mu_0$ + gx }
For this condition the set of equations which we need to consider for obtaining an analytical solution are: 
\begin{align}
4\mu\frac{\partial^2 w}{\partial x^2} + 4\frac{\partial w}{\partial x}\frac{\partial \mu}{\partial x} = -K(u_x - w_x) \\
4\nu\frac{\partial^2 u}{\partial x^2} = K(u_x - w_x)
\end{align}
Replacing unknowns term by term from the previous solutions and given data we have an expression for the difference between $u_x$ and $w_x$: 
%\begin{align*}
\[\boxed{\frac{-4}{K}(\mu_0 + gx )\frac{\partial^2}{\partial x^2}\Bigg( -\frac{T}{4(\nu + \mu)} \Big ( x - \sigma \frac{sinh(\frac{x}{\sigma})}{cosh(\frac{L}{\sigma})}\Big )\Bigg) - \frac{4g}{K}\frac{\partial}{\partial x}\Bigg( -\frac{T}{4(\nu + \mu)} \Big ( x - \sigma \frac{sinh(\frac{x}{\sigma})}{cosh(\frac{L}{\sigma})}\Big )\Bigg)}\]
%\end{align*}
\section*{Mathematica Code}
The code below returns the value of the above expression. f(x) returns the value of $u_x - w_x$
\begin{lstlisting}
sigma[x_] = ((4*(muz + gx)*nu)/(K*(nu+muz+gx)))^0.5
g[x_] = -(Tx/(4*(nu+muz+gx))) + ((T*sigma[x_]*sinh[x/sigma[x]])/(4*(nu+muz+gx)*cosh[L/sigma[x]]))
g'[x_]
g''[x_]
f[x_] = (-4/K)*(muz+gx)*g''[x] + (-4*g/K)*g'[x] 
\end{lstlisting}
\newpage
\section*{Analytical treatment - 12/28/2018}
We assume the trial solution for $u_x$ and $w_x$ for the given problem as: 
\begin{align}
u_x = Ax + Bsinh \Big ( \frac{x}{\sigma} \Big )\\
w_x = Cx + Dsinh \Big ( \frac{x}{\sigma} \Big )
\end{align}
Additionally the extracellular gradient is taken as $\mu = \mu_0 + gx$. The final set of working equations are taken as: 
\[\boxed{\!\begin{aligned}
	&4\mu\frac{\partial^2 w}{\partial x^2} + 4\frac{\partial w}{\partial x}\frac{\partial \mu}{\partial x} = -K(u_x - w_x) \\
	&4\nu\frac{\partial^2 u}{\partial x^2} = K(u_x - w_x)
	\end{aligned}
}
\]
Replacing the values for u and w with that of (44)-(45) we have
\begin{align}
4(\mu_0 + gx)\frac{D}{\sigma^2}sinh(\frac{x}{\sigma}) + 4g(C + \frac{D}{\sigma}cosh(\frac{x}{\sigma})) = -K \Big ( (A - C)x + (B - D)sinh\Big ( \frac{x}{\sigma} \Big )\Big )\\
4\nu\frac{B}{\sigma^2}sinh(\frac{x}{\sigma}) = K \Big ( (A - C)x + (B - D)sinh\Big ( \frac{x}{\sigma} \Big )\Big )
\end{align}
Using boundary conditions on $\epsilon_{exx} = 0$ at x=L (Since we have $\tau_{exx} = 0$ at x=$\pm$L )
\begin{align*}
\frac{\partial w}{\partial x} = 0\\
\therefore \hspace{0.5cm}
C + \frac{D}{\sigma}cosh(\frac{x}{\sigma}) = 0\\
\end{align*}
Plugging the above result in equation (46) and (47) we have: 
\begin{align*}
(\mu_0 + gx)\frac{D}{\sigma^2}sinh(\frac{L}{\sigma}) = -\nu\frac{B}{\sigma^2}sinh(\frac{L}{\sigma})
\end{align*}
\[\boxed{
D = -\frac{\nu}{\mu_0 + gL} B
}
\]
For finding relation between A and C we manipulate equation (47) with the relationship between B and D
\begin{align*}
4\nu\frac{B}{\sigma^2}sinh(\frac{x}{\sigma}) = K \Big ( (A - C)x + (B - D)sinh\Big ( \frac{x}{\sigma} \Big )\Big )\\
\textit{or,} \hspace{0.35cm} 4\nu\frac{B}{\sigma^2}sinh\Big (\frac{x}{\sigma} \Big ) - K (B - D)sinh\Big ( \frac{x}{\sigma} \Big ) = K (A - C)x \\
\textit{or,} \hspace{0.35cm} sinh \Big( \frac{x}{\sigma}\Big ) \Big( \frac{4\nu B}{\sigma^2 } - K(B + \frac{B\nu}{\mu_0 + gL}) \Big ) = K (A - C)x 
\end{align*}
\[\boxed{
A - C = \frac{B}{Kx} sinh \Big( \frac{x}{\sigma}\Big ) \Big( \frac{4\nu }{\sigma^2 } - K(1+ \frac{\nu}{\mu_0 + gL}) \Big )
}
\]
To obtain the values of the unknown parameters B and C, we impose boundary conditions. At the edge (x = $\pm$L) we have, normal stresses $\tau_{ix}$ and $\tau_{ex}$ as zero. Rewriting it, we have,
\begin{align}
C + \frac{D}{\sigma}cosh\Big ( \frac{L}{\sigma} \Big ) = 0\\
A + \frac{B}{\sigma}cosh\Big ( \frac{L}{\sigma} \Big ) = -\frac{T}{4\nu}
\end{align}
Subtracting (49) - (48) we have: 
\begin{align}
(A - C) + (\frac{B - D}{\sigma})cosh\Big ( \frac{L}{\sigma} \Big ) = -\frac{T}{4\nu}
\end{align}
Plugging in A - C and B,D relationship in the equation we have: 
\begin{align*}
\frac{B}{Kx} sinh \Big( \frac{x}{\sigma}\Big ) \Big( \frac{4\nu }{\sigma^2 } - K(1+ \frac{\nu}{\mu_0 + gL}) \Big ) + \frac{1}{\sigma}\Big (B + \frac{B\nu}{\mu_0 + gL}\Big )cosh\Big ( \frac{L}{\sigma} \Big ) = -\frac{T}{4\nu}
\end{align*}
However we have to replace x in the above equation with L as we have been using boundary conditions in (48) - (49). The final form is: (taking out B common in the expression)
\begin{align}
\textbf{B} = -\frac{T}{4\nu} \Bigg ( \frac{1}{sinh \Big( \frac{L}{\sigma}\Big ) \Big ( \frac{4\nu}{\sigma^2 L} - \frac{K}{L} (1+ \frac{\nu}{\mu_0 + gL}) \Big ) + cosh\Big ( \frac{L}{\sigma} \Big ) \Big (  \frac{1}{\sigma} (1+ \frac{\nu}{\mu_0 + gL}) \Big )} \Bigg)
\end{align}
The value of D can be computed using its relationship with B which is $D = -\frac{\nu}{\mu_0 + gL} B$
\begin{align}
\textbf{D} = \frac{T}{4(\mu_0 + gL)} \Bigg ( \frac{1}{sinh \Big( \frac{L}{\sigma}\Big ) \Big ( \frac{4\nu}{\sigma^2 L} - \frac{K}{L} (1+ \frac{\nu}{\mu_0 + gL}) \Big ) + cosh\Big ( \frac{L}{\sigma} \Big ) \Big (  \frac{1}{\sigma} (1+ \frac{\nu}{\mu_0 + gL}) \Big )} \Bigg)
\end{align}
The value of C can be computed from (48) as $C = -\frac{D}{\sigma}cosh\Big ( \frac{L}{\sigma} \Big )$
\begin{align}
\textbf{C} = -\frac{T}{4(\mu_0 + gL)} \Bigg ( \frac{cosh\Big ( \frac{L}{\sigma} \Big )}{sinh \Big( \frac{L}{\sigma}\Big ) \Big ( \frac{4\nu}{\sigma L} - \frac{K \sigma}{L} (1+ \frac{\nu}{\mu_0 + gL}) \Big ) + cosh\Big ( \frac{L}{\sigma} \Big ) \Big ( 1+ \frac{\nu}{\mu_0 + gL} \Big )} \Bigg)
\end{align}
The value of A can be obtained from (49) as $A = -\frac{B}{\sigma}cosh\Big ( \frac{L}{\sigma} \Big ) -\frac{T}{4\nu}$
\begin{align*}
\textbf{A} = -\frac{T}{4(\nu)} \Bigg ( 1 - \frac{cosh\Big ( \frac{L}{\sigma} \Big )}{sinh \Big( \frac{L}{\sigma}\Big ) \Big ( \frac{4\nu}{\sigma L} - \frac{K \sigma}{L} (1+ \frac{\nu}{\mu_0 + gL}) \Big ) + cosh\Big ( \frac{L}{\sigma} \Big ) \Big ( 1+ \frac{\nu}{\mu_0 + gL} \Big )} \Bigg)
\end{align*}
\begin{align}
\therefore \hspace{0.35cm}
\textbf{A} = -\frac{T}{4\nu} \Bigg ( \frac{sinh \Big( \frac{L}{\sigma}\Big ) \Big ( \frac{4\nu}{\sigma L} - \frac{K \sigma}{L} (1+ \frac{\nu}{\mu_0 + gL}) \Big ) + cosh\Big ( \frac{L}{\sigma} \Big ) \Big ( \frac{\nu}{\mu_0 + gL} \Big )}{sinh \Big( \frac{L}{\sigma}\Big ) \Big ( \frac{4\nu}{\sigma L} - \frac{K \sigma}{L} (1+ \frac{\nu}{\mu_0 + gL}) \Big ) + cosh\Big ( \frac{L}{\sigma} \Big ) \Big ( 1+ \frac{\nu}{\mu_0 + gL} \Big )} \Bigg)
\end{align}
Thus the value of of $u_x$ is: 
\begin{align*}
u_x = -\frac{T}{4(\nu)} \Bigg ( \frac{sinh \Big( \frac{L}{\sigma}\Big ) \Big ( \frac{4\nu}{\sigma L} - \frac{K \sigma}{L} (1+ \frac{\nu}{\mu_0 + gL}) \Big ) + cosh\Big ( \frac{L}{\sigma} \Big ) \Big ( \frac{\nu}{\mu_0 + gL} \Big )}{sinh \Big( \frac{L}{\sigma}\Big ) \Big ( \frac{4\nu}{\sigma L} - \frac{K \sigma}{L} (1+ \frac{\nu}{\mu_0 + gL}) \Big ) + cosh\Big ( \frac{L}{\sigma} \Big ) \Big ( 1+ \frac{\nu}{\mu_0 + gL} \Big )} \Bigg)\textbf{x} \\ -\frac{T}{4\nu} \Bigg ( \frac{1}{sinh \Big( \frac{L}{\sigma}\Big ) \Big ( \frac{4\nu}{\sigma^2 L} - \frac{K}{L} (1+ \frac{\nu}{\mu_0 + gL}) \Big ) + cosh\Big ( \frac{L}{\sigma} \Big ) \Big (  \frac{1}{\sigma} (1+ \frac{\nu}{\mu_0 + gL}) \Big )} \Bigg) \textbf{sinh} \Big ( \frac{x}{\sigma} \Big )\\
\end{align*}
The value of $w_x$ is: 
\begin{align*}
w_x = -\frac{T}{4(\mu_0 + gL)} \Bigg ( \frac{cosh\Big ( \frac{L}{\sigma} \Big )}{sinh \Big( \frac{L}{\sigma}\Big ) \Big ( \frac{4\nu}{\sigma L} - \frac{K \sigma}{L} (1+ \frac{\nu}{\mu_0 + gL}) \Big ) + cosh\Big ( \frac{L}{\sigma} \Big ) \Big ( 1+ \frac{\nu}{\mu_0 + gL} \Big )} \Bigg)\textbf{x} + \\
\frac{T}{4(\mu_0 + gL)} \Bigg ( \frac{1}{sinh \Big( \frac{L}{\sigma}\Big ) \Big ( \frac{4\nu}{\sigma^2 L} - \frac{K}{L} (1+ \frac{\nu}{\mu_0 + gL}) \Big ) + cosh\Big ( \frac{L}{\sigma} \Big ) \Big (  \frac{1}{\sigma} (1+ \frac{\nu}{\mu_0 + gL}) \Big )} \Bigg)\textbf{sinh} \Big ( \frac{x}{\sigma} \Big )
\end{align*}
\section*{Issues}
Need some insights for calculating the length constant $\sigma$
\begin{thebibliography}{9}
	\bibitem{roth2012}
	Bradley J Roth
	\textit{The Mechanical Bidomain Model: A Review}
	ISRN Tissue Eng. 2013
	\bibitem{roth2015}
	 Sharma, Kharananda; Al-Asuoad, Nofe; Shillor, Meir; Roth, Bradley J.
	\textit{Intracellular, extracellular, and membrane forces in remodeling and mechanotransduction: The mechanical bidomain model}
	Journal of Coupled Systems and Multiscale Dynamics, Volume 3, Number 3, September 2015, pp. 200-207(8)
\end{thebibliography}
\end{document}
 