\documentclass[a4paper,oneside,11pt]{report}
\usepackage[cm]{fullpage}
\usepackage{lmodern,amsmath,amssymb}
\usepackage{a4wide}
\setlength{\marginparwidth}{3cm}
\setlength{\topmargin}{0cm}
\setlength{\voffset}{0cm}
\setlength{\headsep}{0cm}
\title{Modeling Durotaxis using Mechanical Bidomain model}
\begin{document}
\maketitle
\section*{Analytical treatment}
The equations formulated below consider a gradient in the extracellular stiffness in a 1-d strand of tissue using the mechanical bidomain model. \\

Consider $u_{x}$ and $w_{x}$ as the intra- and extra-cellular displacements of the bidomain layer. x being the principal direction of the one-dimensional tissue strand, T being the tension, p and q being the intra- and extra-cellular pressure contributions arising mostly from hydrostatic forces. \\

The standard equations of the bidomain model are as follows \cite{roth2012}: 
\begin{align}
-\frac{\partial p}{\partial x} + \nu(\frac{\partial^{2}u_x}{\partial x^2} + \frac{\partial^{2}u_x}{\partial y^2} ) + \gamma\frac{\partial^{2}u_x}{\partial x^2} + \frac{\partial T}{\partial x} = K(u_x - w_x) \\
-\frac{\partial q}{\partial x} + \mu(\frac{\partial^{2}w_x}{\partial x^2} + \frac{\partial^{2}w_x}{\partial y^2} ) = -K(u_x - w_x) 
\end{align}   
For this problem we consider:
\begin{itemize} 
\item $\mu = \mu_0 + gx$ 
\item No contributions along the y axis
\item p,q and T are constant and hence their gradients are zero along the x axis
\item Slope of $\mu$ is a constant 
\item K is same for intra- and extra-cellular displacements
\end{itemize}
\begin{align}
    \nu\frac{\partial^{2}u_x}{\partial x^2} + \gamma\frac{\partial^{2}u_x}{\partial x^2} = K(u_x - w_x)  \\
    \mu_0\frac{\partial^{2}w_x}{\partial x^2} + gx\frac{\partial^{2}w_x}{\partial x^2} = -K(u_x - w_x) 
\end{align}
Simplifying the above expression, replacing $u_x$ in equation (2) from (1): 
\begin{align}
   u_x = w_x + \frac{\nu}{K}\frac{\partial^{2}u_x}{\partial x^2} + \frac{\gamma}{K}\frac{\partial^{2}u_x}{\partial x^2} + \gamma\frac{\partial^{2}u_x}{\partial x^2} \\
   \therefore \frac{\mu_0}{K}\frac{\partial^{2}w_x}{\partial x^2} + x\frac{g}{k}\frac{\partial^{2}w_x}{\partial x^2} = - ( \frac{\nu}{K}\frac{\partial^{2}u_x}{\partial x^2} + \frac{\gamma}{K}\frac{\partial^{2}u_x}{\partial x^2} )
\end{align}
Final form the equation can be written as: 
\begin{align}
\frac{\partial^{2}w_x}{\partial x^2} (\frac{\mu_0}{K} + \frac{gx}{K}) + \frac{\partial^{2}u_x}{\partial x^2} (\frac{\nu}{K} + \frac{\gamma}{K} ) = 0
\end{align}
\begin{thebibliography}{9}
	\bibitem{roth2012}
	Bradley J Roth
	\textit{The Mechanical Bidomain Model: A Review}
	ISRN Tissue Eng. 2013
\end{thebibliography}
\end{document}
