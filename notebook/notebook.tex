\documentclass[a4paper,oneside,11pt]{report}
\usepackage[cm]{fullpage}
\usepackage{lmodern,amsmath,amssymb}
\usepackage{a4wide}
\setlength{\marginparwidth}{3cm}
\setlength{\topmargin}{0cm}
\setlength{\voffset}{0cm}
\setlength{\headsep}{0cm}
\title{Modeling Durotaxis using Mechanical Bidomain model}
%\usepackage{etoolbox}
%\preto\equation{\setcounter{equation}{0}}
%\makeatletter
%\pretocmd\start@gather{\setcounter{equation}{0}}{}{}
%\pretocmd\start@align{\setcounter{equation}{0}}{}{}
%\pretocmd\start@multline{\setcounter{equation}{0}}{}{}
%\makeatother
\begin{document}
\maketitle
\section*{Analytical treatment - 10/03/2018}
The equations formulated below consider a gradient in the extracellular stiffness in a 1-d strand of tissue using the mechanical bidomain model. \\

Consider $u_{x}$ and $w_{x}$ as the intra- and extra-cellular displacements of the bidomain layer. x being the principal direction of the one-dimensional tissue strand, T being the tension, p and q being the intra- and extra-cellular pressure contributions arising mostly from hydrostatic forces. \\

The standard equations of the bidomain model are as follows \cite{roth2012}: 
\begin{align}
-\frac{\partial p}{\partial x} + \nu(\frac{\partial^{2}u_x}{\partial x^2} + \frac{\partial^{2}u_x}{\partial y^2} ) + \gamma\frac{\partial^{2}u_x}{\partial x^2} + \frac{\partial T}{\partial x} = K(u_x - w_x) \\
-\frac{\partial q}{\partial x} + \mu(\frac{\partial^{2}w_x}{\partial x^2} + \frac{\partial^{2}w_x}{\partial y^2} ) = -K(u_x - w_x) 
\end{align}   
For this problem we consider:
\begin{itemize} 
\item $\mu = \mu_0 + gx$ 
\item No contributions along the y axis
\item p,q and T are constant and hence their gradients are zero along the x axis
\item Slope of $\mu$ is a constant 
\item K is same for intra- and extra-cellular displacements
\end{itemize}
\begin{align}
    \nu\frac{\partial^{2}u_x}{\partial x^2} + \gamma\frac{\partial^{2}u_x}{\partial x^2} = K(u_x - w_x)  \\
    \mu_0\frac{\partial^{2}w_x}{\partial x^2} + gx\frac{\partial^{2}w_x}{\partial x^2} = -K(u_x - w_x) 
\end{align}
Simplifying the above expression, replacing $u_x$ in equation (2) from (1): 
\begin{align}
   u_x = w_x + \frac{\nu}{K}\frac{\partial^{2}u_x}{\partial x^2} + \frac{\gamma}{K}\frac{\partial^{2}u_x}{\partial x^2} + \gamma\frac{\partial^{2}u_x}{\partial x^2} \\
   \therefore \frac{\mu_0}{K}\frac{\partial^{2}w_x}{\partial x^2} + x\frac{g}{k}\frac{\partial^{2}w_x}{\partial x^2} = - ( \frac{\nu}{K}\frac{\partial^{2}u_x}{\partial x^2} + \frac{\gamma}{K}\frac{\partial^{2}u_x}{\partial x^2} )
\end{align}
Final form the equation can be written as: 
\begin{align}
\frac{\partial^{2}w_x}{\partial x^2} (\frac{\mu_0}{K} + \frac{gx}{K}) + \frac{\partial^{2}u_x}{\partial x^2} (\frac{\nu}{K} + \frac{\gamma}{K} ) = 0
\end{align}
\clearpage
\section*{Analytical treatment - 10/04/2018}
The equations formulated below consider a gradient in the extracellular stiffness in a 1-d strand of tissue using the mechanical bidomain model. \\

Consider $u_{x}$ and $w_{x}$ as the intra- and extra-cellular displacements of the bidomain layer. x being the principal direction of the one-dimensional tissue strand, T being the tension, p and q being the intra- and extra-cellular pressure contributions arising mostly from hydrostatic forces. \\

The intra and extra-cellular stresses arising in the 1-d strand of tissue can be expressed as \cite{roth2015}: 
\begin{align}
\tau_{ix} = -p + 2\nu\epsilon_{ix} + T \\
\tau_{ex} = -q + 2\mu(x)\epsilon_{ex}   
\end{align}
Relationship between the strains and displacement can be written as: 
\begin{align}
\frac{\partial\tau_{ix}}{\partial x} = K (u_x - w_x) \\
\frac{\partial\tau_{ex}}{\partial x} = - K (u_x - w_x) 
\end{align}
Using equation (8) and (9) in (10) and (11) the resulting intra- and extra-cellular equations are: 
\begin{align}
\frac{\partial}{\partial x}(-p + 2\nu\frac{\partial u_x}{\partial x}) = K (u_x - w_x) \\
\frac{\partial}{\partial x}(-q + 2\mu (x) \frac{\partial w_x}{\partial x}) = -K (u_x - w_x)
\end{align}
For the problem we have assumed $\mu = \mu_0 + gx$ , where g is a constant. Rewriting (12) and (13) we have:
\begin{align}
- \frac{\partial p}{\partial x} + 2\nu\frac{\partial^{2}u_x}{\partial x^2} = K (u_x - w_x) \\
- \frac{\partial q}{\partial x} + 2(g\frac{\partial w_x}{\partial x} + \frac{\partial^{2}w_x}{\partial x^2} \mu(x)) = -K(u_x - w_x)
\end{align}
\clearpage
\section*{Analytical treatment - 11/13/2018}
In order to obtain an analytical solution for the problem initially we consider g = 0. The boundary limits of the 1-dimensional problem assume the length of the domain spanning form x=-L to x=+L. The stresses are taken to be zero at each of the boundaries. We begin by implementing a trial solution for $u_x$ and $w_x$.
\begin{align}
u_x = Ax + Bsinh \Big ( \frac{x}{\sigma} \Big )\\
w_x = Cx + Dsinh \Big ( \frac{x}{\sigma} \Big )
\end{align}
For estimating the values of A,B,C,D we assume additionally p=q=0. 
Rewriting (12):
\begin{align}
2\nu\frac{\partial^{2}}{\partial x^2}\Big ( Ax + Bsinh\Big ( \frac{x}{\sigma} \Big ) \Big ) = K \Big ( Ax + Bsinh \Big ( \frac{x}{\sigma} \Big ) - Cx - Dsinh \Big ( \frac{x}{\sigma} \Big )  \Big )\\
2\nu\frac{\partial}{\partial x} \Big ( A + \frac{B}{\sigma}cosh)\Big ( \frac{x}{\sigma} \Big ) = K \Big ((A - C)x + (B - D)sinh\Big ( \frac{x}{\sigma} \Big ) )\\
2\nu\frac{B}{\sigma^2} sinh\Big ( \frac{x}{\sigma} \Big ) = K \Big ( (A - C)x + (B - D)sinh\Big ( \frac{x}{\sigma} \Big )\Big )
\end{align}
and (13)
\begin{align}
2\mu(x)\frac{\partial^{2}}{\partial x^2}\Big ( Cx + Dsinh\Big ( \frac{x}{\sigma} \Big ) \Big ) = -K \Big ( Ax + Bsinh \Big ( \frac{x}{\sigma} \Big ) - Cx - Dsinh \Big ( \frac{x}{\sigma} \Big )  \Big )\\
2\mu(x)\frac{\partial}{\partial x} \Big ( C + \frac{D}{\sigma}cosh)\Big ( \frac{x}{\sigma} \Big ) = -K \Big ((A - C)x + (B - D)sinh\Big ( \frac{x}{\sigma} \Big ) )\\
2\mu(x)\frac{D}{\sigma^2} sinh\Big ( \frac{x}{\sigma} \Big ) = -K \Big ( (A - C)x + (B - D)sinh\Big ( \frac{x}{\sigma} \Big )\Big )
\end{align}
From (20) and (23) we have:
\begin{align}
D = -\frac{\nu}{\mu(x)} B
\end{align}
Comparing coefficients of hyperbolic sine terms we see that (20) and (23) is satisfied only when the coefficient of the linear term is zero. Therefore,
\begin{align}
A = C
\end{align}
Hence the length constant\footnote{This might also be a variable since $\mu$ is a function of x for the problem} $\sigma$ has the value
\begin{equation}
\begin{aligned}
\sigma = \sqrt{\frac{2\nu\mu}{K(\nu + \mu)}}
\end{aligned}
\end{equation}
\begin{thebibliography}{9}
	\bibitem{roth2012}
	Bradley J Roth
	\textit{The Mechanical Bidomain Model: A Review}
	ISRN Tissue Eng. 2013
	
	\bibitem{roth2015}
	 Sharma, Kharananda; Al-Asuoad, Nofe; Shillor, Meir; Roth, Bradley J.
	\textit{Intracellular, extracellular, and membrane forces in remodeling and mechanotransduction: The mechanical bidomain model}
	Journal of Coupled Systems and Multiscale Dynamics, Volume 3, Number 3, September 2015, pp. 200-207(8)
\end{thebibliography}
\end{document}
